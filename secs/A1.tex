\begin{prob}
  Sia $\mathbb{Q}^+$ l'insieme dei razionali strettamente maggiori di $0$. Trovare tutte le funzioni $f: \mathbb{Q}^+ \longrightarrow \mathbb{Q}^+$ t.c. $f(x^2f(y)^2)=f(x)^2f(y)$ per ogni $x, y \in \mathbb{Q}^+$.
\end{prob}

\begin{sol}
  Al solito, sia $P(x,y)$ l'uguaglianza del testo. Vorrei provare $P(0,0)$, ma siamo nei razionali positivi. $P(1,1)$? Viene $f(f(1)^2)=f(1)^3$. $f(x)=x^{3/2}$? A parte che non mi sembra che funzioni, poi non andrebbe nei razionali. Ci sono delle condizioni moltiplicative, quindi è poco probabile che sia una cosa affine generica (o polinomi in generale). $f(x)=x^n$? Vediamo subito che otteniamo la costante $f \equiv 1$. (Nota: sto riportando gli appunti e noto ora che se avessi fatto un check più dettagliato la condizione sull'esponente della $y$ che pensavo essere $2n=n$ si sarebbe invece rivelata essere $2n^2=n$, che avrebbe portato a $n=1/2$. Non so se così sarei arrivato alla soluzione più o meno in fretta, fatto sta che ci sono arrivato comunque; la lezione qui è: state bene attenti ai passaggi che fate!)

  Voglio poter giocare un po': nell'argomento a sinistra posso far comparire un $1$ controllando $P(1/f(y),y)$: $f(1)=f(1/f(y))^2f(y)$. Allora $f(y)=g(y)^2f(1)$, cioè è un quadrato per una qualche costante. Provo a vedere se mi dice qualcosa su $f(1)$, ma ottengo che $f(1)^2$ è un quadrato.

  Iniettività e suriettività? Boh, non sono controllabili con condizioni così "chiuse" (è così che le penso quando non ci sono variabili fuori da $f$).

  Proviamo cose. $P(1,y)$: $f(f(y)^2)=f(1)^2f(y)$. Noi sappiamo anche che $f(f(y)^2)=g(f(y)^2)^2f(1)$. Prese da sole non portano a molto (vi risparmio una o due tautologie sceme), quindi provo un confronto: la stessa cosa scritta in due modi diversi. In questo caso, $f(y)$, così come l'abbiamo appena trovata e poi esplicitando $g(y)$. Quindi, ricordando che in $\mathbb{Q}^+$ possiamo dividere:
  $$f(1)/f(1/f(y))^2=f(y)=f(f(y)^2)/f(1)^2 \implies f(1)^3=f(f(y)^2)f(1/f(y))^2$$
  Ho anche pensato di sostituire nell'uguaglianza originale, ma a occhio direi che peggiorerebbe e basta. $x=y$? Fa schifo. La condizione su $f(y)$ per $y=1$? $f(1)=f(1/f(1))^2f(1) \implies f(1/f(1))^2=1 \implies f(1/f(1))=1$. Abbiamo ottenuto $1 \in \Ima{f}$. Da qui possiamo lavorare un po'. (Consiglio: cercare sempre di ottenere che valori particolari nell'immagine, specialmente zero se si somma e uno se si moltiplica)

  Prendendo $y_1=1/f(1)$ abbiamo che ($P(x,y_1)$): $f(x^2)=f(x)^2$ (*). Potente! $P(x,y)$ si può riscrivere in molti modi, tra cui $f(xf(y))^2=f(x)^2f(y)$. Quindi gli elementi dell'immagine sono tutti dei quadrati. Di più: da (*) otteniamo $f(1)=f(1)^2 \implies f(1)=1$. Proviamo $P(1,y)$: $f(f(y)^2)=f(y)$, sempre per (*) ci dà $f(f(y))^2=f(y)$, quindi sulla sua immagine $f$ agisce come radice quadrata (Visto? È saltata fuori comunque! E continuavo a non accorgermi che sarebbe dovuta uscire prima...). In effetti, sarebbe una bella soluzione, se funzionasse su tutto il dominio. Riusciamo a forzarla? Boh, proviamo.

  Sappiamo che l'immagine è composta da quadrati. Ok, prendiamo $x_0 \in \Ima{f}$. Allora dev'essere $x_0=x_1^2$ e visto che $f$ agisce sull'immagine come la radice quadrata abbiamo che $x_1 \in \Ima{f}$. Per induzione otteniamo $x_0=x_n^{2^n}$. Ragionando ad esempio per assurdo otteniamo che dev'essere per forza $x_0=1$, quindi c'è solo la costante. \\

  Piccola nota finale: abbiamo usato implicitamente un sacco di volte che un quadrato perfetto, in $\mathbb{Q}$, è qualcosa degno di nota. In $\mathbb{R}$ non sarebbe stato possibile.
\end{sol}
